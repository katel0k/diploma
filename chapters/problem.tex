\section{Проблема производительности веб-приложений}
Современные SPA-приложения столкнулись с фундаментальным ограничением: необходимость выполнения значительного объема JavaScript-кода на стороне клиента перед отображением контента. Это приводит к:
\begin{itemize}
    \item Увеличению времени полной загрузки (TTI)
    \item Проблемам с SEO-индексацией
    \item Низкой производительности на мобильных устройствах
\end{itemize}

\section{Server-Side Rendering как решение}
SSR-подход устраняет эти недостатки путем:
\begin{equation}
T_{render} = T_{server} + T_{network} + T_{client\ hydrate}
\end{equation}
где \(T_{server}\) включает генерацию статичного HTML на сервере с последующим кэшированием. Это обеспечивает:
\begin{itemize}
    \item Мгновенную отдачу контента
    \item Улучшение Core Web Vitals
    \item Полную SEO-совместимость
\end{itemize}