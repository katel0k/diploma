\section{Архитектурные принципы React}
React.js доминирует в веб-разработке благодаря:
\begin{itemize}
    \item Компонентной модели на основе чистых функций
    \item JSX как декларативному языку разметки
    \item Одностороннему потоку данных
\end{itemize}

\section{Выбор OCaml для SSR}
Обоснование использования OCaml:
\begin{itemize}
    \item \textbf{Типобезопасность}: статическая проверка шаблонов
    \item \textbf{Функциональная парадигма}: естественная реализация React-подобных систем
    \item \textbf{Производительность}: нативные бинарники через компиляцию в C
    \item \textbf{Экосистема}: наличие фреймворка Dream для веб-разработки
\end{itemize}

\begin{lstlisting}[caption=Пример компонента на OCaml]
let greet name = 
  Dream.html (Dream.eml %s{<h1>Hello, <%s name %>!</h1>})
\end{lstlisting}