% \section{Критерии оценки}
% \begin{enumerate}
%     \item \textbf{Производительность}: время генерации 10,000 элементов
%     \item \textbf{Потребление памяти}: аллокации/деаллокации
%     \item \textbf{Интеграция с инструментами}: bisect-ppx
%     \item \textbf{Эргономичность разработки}
% \end{enumerate}

% \section{Инструментарий}
% \begin{itemize}
%     \item \textbf{Тестовый стенд}: Intel Xeon E5-2680, 32GB RAM
%     \item \textbf{Мониторинг}: Valgrind Massif, Perf
%     \item \textbf{Тестовый сценарий}: 
%         \begin{lstlisting}
%         let test_template = 
%           List.map (fun i -> div [id i] [txt (string_of_int i)]) 
%           (List.init 10000 Fun.id)
%         \end{lstlisting}
% \end{itemize}

для того чтобы сравнить фреймворки была сделана небольшая тестовая страница. ключевыми параметрами для качественного сравнения рассматривалось субъективное удобство написания страницы, наличие поддержки со стороны редакторов кода, результирующий код после работы препроцессоров а также сравнивалось качество проверки кода.

Была также задача сравнить фреймворки с точки зрения производительности. Для этого был созданн бенчмарк - простой тест создания страницы с большим количеством одинаковых эелмнтов.

Уникальная чассть работы заключалась также в том, что потребовалось для каждого варианта препроцессинга написать свою реализацию кода для тестирования.

% // TODO: упомянуть возникшую проблему с тем как я генерировал страницу?
% // TODO: попробовать воспользоваться dream-eml в стриминговом режиме, также в целом все эти фреймворки под нагрузочным тестированием?



Для тестирования был выбран логарифмический масштаб измерений чтобы проанализировать степенную зависимость. Предполагается что время работы работы программы будет составлять $\mathcal{O}(n)$, где n - количество сгенерированных элемнтов.

Время работы замерялось посредством внутренних инструемнтов языка. 




% Результаты представлены на графике 



