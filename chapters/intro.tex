Актуальность исследования обусловлена растущими требованиями к производительности веб-приложений. Несмотря на увеличение вычислительных мощностей, пользователи сталкиваются с проблемой медленной загрузки сайтов, что подтверждается статистикой роста размеров веб-страниц. 

\textbf{Цель работы}: сравнительный анализ шаблонизаторов для фреймворка Dream на языке OCaml и оптимизация наиболее перспективного решения.

\textbf{Задачи}:
\begin{enumerate}
    \item Анализ современных подходов к серверному рендерингу (SSR)
    \item Сравнение характеристик шаблонизаторов: ocaml-mlx, tyxml, dream-html, dream-eml
    \item Разработка методики тестирования производительности
    \item Оптимизация механизма рендеринга dream-eml
    \item Валидация результатов через нагрузочное тестирование
\end{enumerate}

\textbf{Научная новизна}: предложен механизм снижения аллокаций памяти в шаблонизаторах с сохранением чистоты функций.