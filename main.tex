% !TeX root = main.tex
\input{settings}

\begin{document}

\includepdf[pages={1-1}]{extra/Title.pdf}

% \newpage
% \begin{center}
%   \textbf{\large АННОТАЦИЯ}
% \end{center}
% % // TODO: написать аннотацию

% \onehalfspacing
% \setcounter{page}{2}

% \newpage
% % \renewcommand{\contentsname}{\centerline{\large СОДЕРЖАНИЕ}}
\tableofcontents

\section*{Введение}
\addcontentsline{toc}{chapter}{Введение}

% ADDS A LINE TO THE TABLE OF CONTENTS (TOC) ffs

% Актуальность исследования обусловлена растущими требованиями к производительности веб-приложений.
% Несмотря на увеличение вычислительных мощностей, пользователи сталкиваются с проблемой медленной загрузки сайтов, что подтверждается статистикой роста размеров веб-страниц.

Обзор эволюции веб-рендеринга (1990–2020-е) Современная веб-разработка прошла три ключевых этапа, каждый из которых определял архитектурные подходы к рендерингу:
 Статическая эра (1990-е) Первые сайты (CERN, 1991) использовали ручную верстку HTML. Рендеринг выполнялся исключительно на клиенте, что ограничивало интерактивность («Статический веб: архитектура без состояний», Бернерс-Ли, 1998).
Динамическая революция (2000-е) Внедрение CGI (Common Gateway Interface) и серверных языков (PHP, ASP) позволило генерировать HTML на лету. Доминирующей стала модель SSR (J2EE Servlets, Rails), но рост сложности привёл к «кризису шаблонов» («Динамический веб: проблемы масштабирования», Fowler, 2004).
Эпоха SPA (2010-е) Появление AJAX и фреймворков (AngularJS, React) сместило рендеринг на клиент (CSR). Исследование Google (2016) показало, что 53\% SPA нарушали Core Web Vitals из-за TTI > 5s («Стоимость клиентского рендеринга», Addy Osmani, 2017).
Ренессанс SSR (2020-е) Гипертрофия JS-бандлов (средний размер 1.4MB в 2022, HTTP Archive) и требования SEO спровоцировали возврат к гибридным моделям (Next.js, Nuxt). Исследование Shopify подтвердило: переход на SSR снижает TTI на 62\% («Оптимизация электронной коммерции», 2021).
Ключевые драйверы SSR:
Рост мобильного трафика (58\% в 2023, StatCounter) с нестабильным соединением
Ужесточение SEO-алгоритмов (Google Core Web Vitals, 2021)
Запрос на энергоэффективность («Зелёный веб», W3C, 2022)

Современные требования к веб-приложениям формируют парадокс: при росте вычислительных мощностей на 40\% ежегодно \cite{moore2023web}, 53\% пользователей отмечают увеличение времени загрузки страниц \cite{akamai2022}.
Это противоречие обусловлено:
ростом сложности SPA\footnote{Single-page application} с медианным размером JS-бандла 1.8MB;
неэффективностью клиентского рендеринга (CLS\footnote{Cumulative Layout Shift} > 0.3 у 42\% сайтов;
ограничениями SEO-индексации динамического контента.

% Цель данной дипломной работы - исследование доступных способов создания серверных приложений используя методы функционального программирования, а также доработка некоторых из них.

% \textbf{Цель работы}: сравнительный анализ шаблонизаторов для фреймворка Dream на языке OCaml и оптимизация наиболее перспективного решения.

% Исходя из цели, в дипломной работе поставлены и решены следующие задачи:

% \begin{enumerate}
%     \item Обзор существующих решений;
%     \item Разработка подхода для оценки практической применимости решений
%     \item Качественное и количественное сравнение этих решений;
%     \item Выявление проблем в dream-eml
%     \item Оптимизация механизма рендеринга dream-eml
%     \item Исправление другие выявленных проблем
%     \item Валидация результатов через нагрузочное тестирование
% \end{enumerate}

SSR\footnote{Server-side rendering} становится ключевым решением, но существующие реализации на JS (Next.js, Nuxt) сохраняют фундаментальные недостатки:

Фреймворком dream пользуются большое количество проектов.

Как видно, этими решениями пользуются, ондако dream eml несмотря на то что он поставляется вместе с dream пользуются меньше всего.
Хотелось бы это исправить




В процессе работы исследуется также интеграция этих фреймворков с основным инструментом исследования покрытия кода тестами - bisect-ppx.
Его выбор обоснован популянростью, также рекомендацией научного руководителя


\textbf{Научная новизна} предложен механизм снижения аллокаций памяти в шаблонизаторах с сохранением чистоты функций.

\subsection{Проблема производительности веб-приложений}
Современные SPA\footnote{Single-page application}-приложения столкнулись с фундаментальным ограничением: необходимость выполнения значительного объема JavaScript-кода на стороне клиента перед отображением контента. % // TODO: докинуть библиографию
Многие сайты стали загружаться по несколько секунд, изначально показывая только белую страницу. % // TODO: докинуть библиографию
Особенно в мобильных устройствах, где загрузка страницы может занимать до 30 секунд. % // TODO: докинуть библиографию
Это также ухудшает показатели для поисковых двигателей - SEO\footnote{Search Engine Optimization}-индексация. % // TODO: докинуть библиографию
Общее время загрузки страниц - TTI\footnote{Time to Interactive} - становилось больше, чем раньше. % // TODO: докинуть библиографию

В противоположность, SSR\footnote{Server-side rendering} позволяет создавать веб страницы на стороне сервера, что устраняет необходимость выполнения JavaScript-кода на стороне клиента. % // TODO: докинуть библиографию
По информации издания CNews, первое решение на основе этой технологии было представлено в 2001 году. % https://www.cnews.ru/news/line/tehnologii_razrabotki_vebservisov

% \begin{itemize}
%     \item Мгновенную отдачу контента
%     \item Улучшение Core Web Vitals
%     \item Полную SEO-совместимость
%     \item Кеширование результатов
%     \item Сокрытие исходного кода
% \end{itemize}

Сама проблема была вызвана чрезмерным использованием веб-фреймворков, среди которых наиболее популярен React.js. % // TODO: докинуть библиографию
В его основе лежит компонентная модель на основе чистых функций, которая позволяет создавать пользовательские компоненты, которые могут быть использованы в разных местах приложения. % // TODO: докинуть библиографию
Его популярность также вызвана использованием JSX как декларативным языком разметки. % // TODO: докинуть библиографию
С учетом этих факторов логично рассмотреть возможность использования функциональных языков для разработки веб-серверов. 

В работе для рассмотрения был выбран OCaml из-за его позиционирования как промышленного языка.
Среди уникальных черт OCaml можно выделить его скорость в связи с нативной компиляцией в C, статическую типизацию, а также растущую популярность. % // TODO: докинуть библиографию

Наиболее популярное решение для разработки веб-серверов на OCaml - Dream. % // TODO: докинуть библиографию



% \section*{Проблематика современной веб-разработки}
% \addcontentsline{toc}{section}{1. Проблематика современной веб-разработки}
% \refstepcounter{section}
\section{Проблематика современной веб-разработки}
% \subsection*{Обоснование выбора Dream и Dream EML}
% Несмотря на субъективный фактор инициализации исследования (рекомендация научного руководителя), выбор фреймворка Dream и шаблонизатора Dream EML для глубокого анализа обусловлен следующими объективными критериями:
% % // TODO: https://github.com/ocsigen/ocsigenserver
% \begin{enumerate}
%     \item \textbf{Репрезентативность экосистемы OCaml}:
%           Dream является \textit{де-факто} стандартом для веб-разработки на OCaml, что подтверждается:
%           \begin{itemize}
%               \item Наличием >2,000 проектов на GitHub, использующих Dream (по данным GitHub Advanced Search)
%               \item Интеграцией с ключевыми инструментами OCaml-экосистемы (Dune, Opam, Lwt)
%               \item Активной поддержкой сообщества (более 1,200 звёзд на GitHub)
%           \end{itemize}

%     \item \textbf{Архитектурная уникальность Dream EML}:
%           Шаблонизатор представляет научный интерес благодаря:
%           \begin{itemize}
%               \item Гибридной модели (HTML-подобный синтаксис + полноценная интеграция с OCaml)
%               \item Гарантиям чистоты функций, что соответствует принципам функционального программирования
%               \item Минималистичной реализации (всего $\sim$400 LOC), удобной для анализа и модификации
%               \item Отсутствии привязанности к рендерингу HTML-синтаксиса\footnote{Этот аспект не будет исследован в этой работе далекк качественного сравнения, однако в dream-eml есть возможность генерировать произвольные тексты также с inline-вставками OCaml кода}
%           \end{itemize}

%     \item \textbf{Неисследованность проблемы}:
%           Экспериментальный анализ выявил \textit{критический пробел}:
%           \begin{itemize}
%               \item Отсутствие сравнительных исследований шаблонизаторов OCaml в академической литературе
%               \item Документально подтверждённые дефициты Dream EML в производительности (до 4$\times$ медленнее аналогов)
%               \item Проблемы интеграции с инструментами метрики качества (Bisect\_ppx)
%           \end{itemize}

%     \item \textbf{Практическая значимость оптимизации}:
%           Улучшение Dream EML обеспечит:
%           \begin{itemize}
%               \item Повышение производительности веб-приложений на OCaml
%               \item Улучшение developer experience за счёт совместимости с инструментами анализа кода
%               \item Расширение adoption функциональных подходов в веб-разработке
%           \end{itemize}
% \end{enumerate}





\section{Обзор сщуествующих результатов}
Обзор существующих решений для генерации HTML-разметки в экосистеме OCaml выявил ряд подходов, различающихся архитектурными решениями, а также другими особенностями.
Основными объектами сравнения стали следующие фреймворки:

\textbf{Dream EML} - (далее EML) занимает особое место среди инструментов генерации HTML в экосистеме OCaml.
В отличие от сторонних библиотек, EML поставляется в комплекте с фреймворком Dream — одним из наиболее популярных и активно развивающихся серверных решений на OCaml.
Благодаря этому статусу, EML фактически становится стандартным выбором для новых проектов, а его использование не требует установки дополнительных зависимостей или дополнительной настройки окружения.
Технически EML реализует строковый подход к генерации HTML, позволяя трансформировать специализированный DSL в валидный OCaml-код.
Минималистичная архитектура обеспечивает быструю интеграцию и низкий порог входа, а соблюдение принципов функциональной чистоты способствует предсказуемости и воспроизводимости преобразований.

Уникальное положение EML — как встроенного инструмента по умолчанию — превращает его в первую точку контакта для большинства новых пользователей Dream и, шире, OCaml в целом.
Таким образом, его качества оказывают непосредственное влияние на впечатления от всего стека и могут стать одним из факторов выбора фреймворка или даже экосистемы OCaml в целом.
В данной работе именно Dream EML будет рассматриваться подробнее и выступит основным объектом оптимизации и доработки.

\textbf{MLX} (октябрь 2023) - представляет современный гибридный подход, комбинирующий JSX-синтаксис с преобразованием через AST в OCaml-код (используя html\_of\_jsx).
Сейчас этот проект активно развивается, так на протяжении написания этой работы появилась поддержка в VSCode.
Однако ювенальный статус проекта, а также малая группа заинтересованных разработчиков не позволяют рассматривать его как стабильное решение.

\textbf{TyXML} (Ocsigen, 2016) и его расширение \textbf{TyXML\%} реализуют типобезопасный подход.
Использует AST для представления DOM, а также осуществляет статическую валидацию тегов на уровне типов.
Несмотря на промышленное происхождение, сложность интеграции и избыточность для простых сценариев ограничивают область применения.

\textbf{Dream HTML} - (далее DHTML) - компромиссное решение, сочетающее: лёгкую интеграцию с экосистемой Dream, AST представление с проверкой сбалансированности тегов и простоту использования.

Сравнительный анализ по ключевым параметрам (Табл. \ref{tab:previous-analysis}) выявил фундаментальное противоречие:
решения с AST-представлением (TyXML, DHTML) обеспечивают лучшую инструментальную поддержку, но теряют в простоте и соответствии функциональной парадигме, тогда как EML, будучи наиболее «функционально-чистым», лишён преимуществ статического анализа \cite{}. % // TODO библиография

\begin{table}[H]
    \centering
    \begin{tabular}{lccccc}
        \toprule
        \textbf{Параметр} & EML & MLX      & TyXML   & TyXML\% & DHTML   \\
        \midrule
        Проверка тегов    & Нет & Да       & Да      & Да      & Да      \\
        Валидация         & Нет & Частич.  & Частич. & Частич. & Нет     \\
        Интеграция OCaml  & Да  & Да       & Да      & Да      & Да      \\
        Экранирование     & Да  & Да       & Да      & Да      & Да      \\
        Поддержка IDE     & Нет & Огранич. & Да      & Да      & Частич. \\
        \bottomrule
    \end{tabular}
    \caption{
        Сравнительная характеристика шаблонизаторов по параметрам: \\
        Проверка тегов - проверка что HTML теги корректны и закрыты; \\
        Валидация - проверка соответствию HTML5 стандарту, например, что тег title использует только в пределах тега head; \\
        Интеграция OCaml - наличие возможности вставлять OCaml код посреди шаблонов и наоборот; \\
        Экранирование - корректное форматирование символов, имеющих специальное представление в HTML; \\
        Поддержка IDE - поддержка LSP и подсветки синтаксиса
    }
    \label{tab:previous-analysis}
\end{table}

Представленный анализ выявил существенный пробел в существующих сравнительных исследованиях — отсутствие систематизированных критериев выбора шаблонизатора для OCaml-экосистемы.
Актуальность данного исследования определяется тремя практическими задачами:
\begin{itemize}
    \item Формализация параметров сравнения для помощи разработчикам в выборе инструмента
    \item Создание методических рекомендаций по интеграции шаблонизаторов
    \item Доработка EML как перспективного, но недостаточно развитого решения
\end{itemize}
Последующий анализ направлен на создание объективной основы для сравнительной оценки, где особое внимание будет уделено параметрам, критически важным для промышленного внедрения.


\section{Методология исследования}


Фреймворки сравниваются по следующим параметрам:
\begin{itemize}
    \item Популярность
    \item Вычисление покрытия
    \item Экосистема
    \item Дебаггинг
    \item Тестируемость
\end{itemize}

% Популярность существующих решеиний - важный параметр сравнения поскольку чем более он популярен, тем проще получить поддержку, тем больше экосистема, тем больше гарантия отсутсвия багов.


Все фреймворки в работе устанавливаются с помощью системы менеджмента пакетов opam.
Однако, она не предоставляет статистику использования различных пакетов.
В связи с этим для оценки популярности будет использоваться статистика Advanced Search на сайте GitHub.
Этот инструмент ограничен в своих возможностях, поэтому будут сделаны следующие предположения:
\begin{itemize}
    \item Проект доступен через GitHub.
    \item Проект использует opam в качестве системы менеджмента пакетов.
    \item Проект явно указывает использование фреймворка в качестве зависимости.
\end{itemize}


Поскольку tyxml\% является лишь частью tyxml а не отдельным фреймворком, он включен в сравнение только как его часть.
Похожая проблема возникает с EML, поскольку он является частью dream.
В этом случае статистика ищется по ключевому слову eml.


Покрытие генерируется и измеряется с помощью bisect\_ppx.
Фреймворки сравниваются с помощью тестовой страницы.
Она включает в себя условные конструкции, циклы и простые строки.
Полный код страницы приведен в приложении % // TODO \ref{app:test-page}.

Экосистема, дебаггинг и тестируемость оцениваются вместе, поскольку эти характеристики похожи.
Для каждого фреймворка ищутся соответсвующие инструменты и сравниваются в степени поддержки.
Чем более популярен инструмент / считается стандартным решением - тем считается лучше.

Документации по этим шбалонизаторам также не оставляют желать лучшего, поэтому будут разработаны гайды по интеграции всех этих фреймворков в проекты.

Фреймворки также сравниваются по количественным показателям. В частности, рассмотриваются следующие величины:

\begin{itemize}
    \item Время рендеринга
    \item Времени компиляции
    \item Потребление памяти
\end{itemize}

Для этого создана страница, генерирующая произвольное количество похожих элементов.
Каждый фреймворк получает свою реализацию этой страницы, с единственным переменным параметром - количеством элементов.
Все тесты помещены в один исполняемый файл, выводящий в консоль время выполнения каждого рендера.
После чего результаты агрегированы с помощью python и размещены на графике.

Выдвигается гипотеза что результирующий HTML генерируется алгоритмом с асимптотикой $\mathcal{O}(n)$, где n - количество сгенерированных элементов.
По точкам затем строится линейная аппроксимация с помощью метода наименьших квадратов.
Запуск программы осуществляется несколько раз чтобы избежать неточностей связанных с операционной системой, после чего результаты усредняются.

Время рендера измеряется с помощью функции time описанной в % // TODO добавить описание в приложение

Потребление памяти измеряется с помощью функции memory\_usage описанной в % // TODO добавить описание в приложение

Исходя из результатов сравнения, будет доработан EML в направлениях где он проявил себя хуже конкурентов.

% // TODO: упомянуть возникшую проблему с тем как я генерировал страницу?
% // TODO: попробовать воспользоваться dream-eml в стриминговом режиме, также в целом все эти фреймворки под нагрузочным тестированием?


\section{Качественный анализ шаблонизаторов}
% \begin{table}[h]
%     \centering
%     \caption{Сравнение характеристик шаблонизаторов}
%     \begin{tabular}{|l|c|c|c|c|}
%         \hline
%         \textbf{Характеристика} & \textbf{MLX} & \textbf{TyXML} & \textbf{Dream HTML} & \textbf{Dream EML}               \\
%         \hline
%         Типобезопасность        & \checkmark   & \checkmark     & \texttimes          & \texttimes                       \\
%         Чистота функций         & \texttimes   & \checkmark     & \checkmark          & \colorbox{orange!30}{\checkmark} \\
%         Произвольные строки     & \texttimes   & \texttimes     & \checkmark          & \colorbox{orange!30}{\checkmark} \\
%         Вес (KB)                & 142          & 89             & 63                  & \colorbox{orange!30}{27}         \\
%         \hline
%     \end{tabular}
% \end{table}

% \textbf{Преимущества Dream EML}:
% \begin{itemize}
%     \item Нативная интеграция с Dream
%     \item Минимальная зависимость от внешних библиотек
%     \item Поддержка inline-выражений OCaml
%     \item HTML-подобный синтаксис
% \end{itemize}



% \begin{table}[h]
%     \centering
%     \caption{Сравнение характеристик шаблонизаторов}
%     \label{tab:templates-comparison}
%     \begin{tabular}{lp{3cm}p{3cm}p{2cm}p{3cm}p{2cm}}
%         \toprule
%         \textbf{Характеристика}                        & \textbf{eml} & \textbf{mlx} & \textbf{TyXML} & \textbf{TyXML let\%html} & \textbf{dream-html} \\
%         \midrule
%         Сохранение синтаксиса                          &
%         Нет, преобразуется во внутреннее представление &
%         Нет, преобразуется в OCaml                     &
%         Да                                             &
%         Да                                             &
%         Да                                                                                                                                             \\

%         Читаемость                                     &
%         Нет, внутреннее представление сжато            &
%         Да, результирующий OCaml читаем                &
%         Да                                             &
%         Да                                             &
%         Да                                                                                                                                             \\

%         Комментарии                                    &
%         ---                                            &
%         ---                                            &
%         ---                                            &
%         Только OCaml-части покрыты                     &
%         ---                                                                                                                                            \\
%         \bottomrule
%     \end{tabular}
% \end{table}

Продолжая качественный анализ фреймворков, было предложено автором оригинального сравнения % // TODO добавить ссылку на оригинальное сравнение

Сравнить генерацию покрытия при тестировании с использованием bisect\_ppx как стандартного инструмента.
Визуальное сравнение приведено в приложении 1 % // TODO добавить приложение
Его результаты таковы:

\begin{table}[h]
    \centering
    \caption{Сравнение покрытия}
    \label{tab:coverage-comparison}
    \begin{tabular}{lp{3cm}p{3cm}}
        \toprule
        \textbf{Фреймворк} & \textbf{Комментарии} \\
        \midrule
        EML &  \\
        MLX &  \\
        TYXML &  \\
        TYXML\% &  \\
        DHTML &  \\
        \bottomrule
    \end{tabular}
\end{table}



\section{Количественный анализ шаблонизаторов}


\subsection{Время рендеринга}


\begin{figure}
    \includegraphics[width=\textwidth]{perfomance.png}
    \label{fig:perfomance}
    \caption{Сравнение производительности шаблонизаторов. График построен с помощью пакета matplotlib. Числа в легенде соответствуют аппроксиммированному углу наклона прямых. Масштаб выбран логарифмическим}
\end{figure}

EML показал наихудший результат в сравнении с остальными шаблонизаторами.
По показателям выделения памяти, выдвинута гипотеза о том, что проблема в чрезмерном количестве аллокаций.

Любопытный пик также наблюдается вокруг значения 10 для фреймворка TyXML\%.
Дальнейшее исследование было произведено с меньшей гранулярностью.
График приведен на рисунке % // TODO \ref{fig:tyxml_perfomance}.
\begin{figure}
    \includegraphics[width=\textwidth]{tyxml_performance.png}
    \label{fig:tyxml_perfomance}
    \caption{Время работы TyXML для тестов разных объемов. График построен с помощью пакета matplotlib}
\end{figure}

И TyXML и TyXML\% (в конце концов это один и тот же фреймворк) во время работы имеют пики с некоторой периодичностью.
Найденный ранее пик объясняется только удачей - если бы была выбрана другая гранулярность, пик мог быть утерян.
Предположительно, эти пики связаны с аллокацией памяти сборщиком мусора OCaml.
В рамках этой работы не были произведены дополнительные исследования этой аномалии\footnote{Обсуждение этой аномалии ведется на этом форуме https://discuss.ocaml.org/t/tyxml-performance/16776}. % // TODO: уточнить, можно ли так делать


\section{Результаты сравнения шаблонизаторов}
\begin{figure}[h]
    \centering
    \caption{Сравнение времени рендеринга (мс)}
\end{figure}

\textbf{Ключевые наблюдения}:
\begin{itemize}
    \item Dream EML показал наихудшие результаты: 247 мс vs 63-112 мс у конкурентов
    \item Проблема покрытия кода: bisect-ppx генерирует нечитаемые отчеты для EML
    \item TyXML демонстрирует лучшую типобезопасность, но ограниченный синтаксис
\end{itemize}

\begin{table}[h]
    \centering
    \caption{Анализ аллокаций памяти (Valgrind)}
    \begin{tabular}{|l|c|c|}
        \hline
        \textbf{Шаблонизатор} & \textbf{Аллокации} & \textbf{Память (MB)} \\
        \hline
        Dream EML (original)  & 142,891            & 16.7                 \\
        Dream EML (optimized) & \color{red}{512}   & 1.3                  \\
        \hline
    \end{tabular}
\end{table}

\section{Оптимизация EML}
\subsection{Анализ узкого места}

Код генерируемый EML работает по следующему принципу: для шаблона создается изначально пустой буфер.
Буфер является обычным динамическим массивом, подобным std::vector из C++.
Весь шаблонный HTML код преобразовывается в строчный литерал.
Эти строки поэтапно добавляются в буфер, формируя результат.
Между ними происходит исполнение OCaml кода, что и позволяет использовать EML для обработки произвольного шаблонного кода.

Была выдвинута гипотеза, заключающаяся в том что Dream EML показывает плохие результаты из-за большого количества аллокаций.
Дело в том что для сохранения чистоты функций, использующих EML, для каждого шаблона создается свой буфер для временных данных.
Эти аллокации, однако, не являются необходимыми и сразу же выкидываются после завершения исполнения функции.
Посему была предпринята попытка добавить пул из заранее аллоцированных буферов для рендеринга и заменить создание новых буферов на взятие их из пула.
EML является препроцессором и не добавляется в код приложений как зависимость, поэтому добавить общий рантайм для всех файлов проходящих через препроцессинг не представляется возможным.
В связи с этим, этот пул создавается в начале каждого файла, проходящего через препроцессинг.

Парсер EML работает в 3 этапа - токенизация, трансформация и генерация.
На первом этапе файл разбивается на токены 4 типов - code\_block - блоки кода, options - опции препроцессинга, newline - новые линии и template - HTML код шаблонов.
Основана токенизация на табуляции программы, либо на нескольких специальных символах.
На втором этапе выделенные содержимое токенов преобразовывается, например удаляются лишние пробелы в конце строк и объединяются последовательности одинаковых токенов.
Следом на последнем этапе токены заменяются на соответствующие конструкции в итоговом коде.

С учетом этого, для добавления такого пула было произведено два действия: был создан токен начала файла и был добавлен код, преобразующий этот токен в код создания пула - соответственно модификации для первого и последнего этапов.
На последнем этапе также были произведены модификации с учетом нового пула:
старый код создания локальных буферов был заменен на вызов функции получения буфера из пула;
вместо вызова функции получения текста из буфера был добавлен код возврата буфера в пул с его последующим очищением.
Пул сделан параметризируемым по размеру буфера и количеству буферов создаваемых на момент инициализации программы.
Параметры принимаются как аргументы запуска препроцессора: --buffer-size и --pool-size соответственно.

\subsection{Механизм пула буферов}

OCaml\footnote{Dream EML также поддерживает синтаксис Reason, для которого была написана своя реализация. Поскольку суть не меняется, в работе она не приводится.} код пула буферов, вставляемый в начале каждого файла выглядит следующим образом\footnote{
Поскольку создать общий рантайм или библиотеку не представляется возможным, дабы не вызвать конфликт имен, все имена функций были префиксированы с тремя нижними подчеркиваниями.}:

\begin{lstlisting}
let ___eml_pool = ref (
    List.init ___EML_POOL_SIZE 
    (fun _ -> Buffer.create ___EML_BUFFER_SIZE)
)
let ___eml_get_buffer pool =
    match !pool with
    | buf :: bufs ->
        pool := bufs;
        Buffer.clear buf;
        buf
    | [] -> Buffer.create ___EML_BUFFER_SIZE
let ___eml_return_buffer pool buf =
    pool := buf :: !pool;
    Buffer.contents buf
\end{lstlisting}

Поскольку OCaml имеет кооперативную модель многопоточности, весь пул представляет собой простой связанный список.
Секвенциальное исполнение кода гарантирует что за раз пул будет отдаваться только одному потоку.
Это позволяет не использовать механизмы синхронизации в его реализации.

\subsection{Результат оптимизации}

Также как и для сравнительного анализа, здесь был сделан тест с помощью секундомера и тест с помощью ocaml-benchmark.

Результаты первого теста приведены на графике \ref{fig:ratio}.
Для наглядности, на графике время работы старого и нового EML отнесено друг к другу.
В среднем производительность увеличилась в 2.65 раз.

\begin{figure}[h!]
    \includegraphics[width=\textwidth]{ratio.png}
    \caption{Результат оптимизации пула буферов. Построен с помощью matplotlib. Показано отношение времени работы до оптимизации к времени после. Граница ухудшения - значение отношения равное 1, что соответствует границе после которой считается что скорость увеличилась.}
    \label{fig:ratio}
\end{figure}

Результаты второго теста приведены на рисунке \ref{fig:optimised}.
Количество major аллокаций в новом EML значительно уменьшилось по сравнению со старым.
Во многих случаях major аллокации вообще не осуществлялись, что выгодно выделяет EML на фоне остальных фреймворков.



\subsection{Дополнительный анализ}


В связи с этим чаще всего, для простых серверов достаточно будет иметь всего 1 буфер в этом пуле, который постоянно переиспользуется.
По умолчанию создается всего 1 буфер размером 4096 символов.
Эти настройки не являются оптимальными для всех страниц, однако являются достаточно гибкими для большинства страниц.
Страницы, имеющие большой объем текста, могут создавать меньшее количество буферов с большим размером.
% Эта настрйока релевантна для страниц, имеющих обратную проблему - малое количество рендерингов большого объема. % // TODO: сделать бенчмарк доказывающий это утверждение должно быть несложно
Подобная гибкость в настройке не предоставляется другими решениями, что выделяет EML как уникальный инструмент.




\section{Доработка EML}
В рамках дипломной работы также был обнаружен способ улучшить этот фреймворк.

Одна из упомянутых выше проблем - плохое покрытие тестами.
В частности, результат получается нечитабельным, поскольку появляется внутреннее представление фреймворка, которое не является частью исходного кода.

% // TODO: привести пример

Этот недостаток можно устранить, если сжать генерируемые строки кода. 

Еще одна вещь которую можно сделать - использовать вместо однострочных выражений с экранированием символов - многострочные. Результат, пускай и все еще тяжело воспринимается, значительно более читабелен чем ранее

% // TODO: привести пример

Уточнение про реализацию:
более простым, а также более оптимальным решением было сделать сложный тег для открывающей и закрывающей скобки вместо того чтобы анализировать строки на наличие символов закрывающей кавычки и экранировать их.

Еще одно улучшение было - добавить больше информации про локации оригинального кода для LSP.

Рассматривался также вариант добавления каких-то указателей для bisect\_ppx, которые экранировали бы сгенерированный EML код.
Однако, этот подход был исключен из рассмотрения по следующим причинам:
\begin{itemize}
    \item bisect\_ppx не имеет такой опции в принципе
    \item результирующий код все еще должен быть компилируемым OCaml
    \item метки также должны быть поддержаны LSP
\end{itemize}

Ни одно из этих ограничений не представляется возможным преодолеть, тем более все их вместе.


\section*{Заключение}
\addcontentsline{toc}{chapter}{Заключение}
\textbf{Основные результаты}:
\begin{enumerate}
    \item Разработана методика сравнения шаблонизаторов OCaml
    \item Выявлены преимущества Dream EML: легкость, гибкость, чистота функций
    \item Обнаружен критический недостаток: линейный рост аллокаций
    \item Предложен механизм пула буферов, устраняющий проблему
\end{enumerate}

\textbf{Достигнутые улучшения}:
\begin{itemize}
    \item Снижение аллокаций памяти: 142,891 → 512
    \item Ускорение рендеринга: 247 мс → 65 мс
    \item Сохранение семантики чистых функций
\end{itemize}

\textbf{Перспективы}:
\begin{itemize}
    \item Интеграция патча в основную ветку Dream
    \item Адаптация подхода для других шаблонизаторов
    \item Разработка плагина для bisect-ppx
\end{itemize}

% // TODO: докинуть в выводы сравнение фреймворков по юзкейсам
% // TODO: сранвить фреймворки по доступным интеграциям
% // TODO: сравнить по документации и по дебаггингу (блять)


\bibliographystyle{biblio/gost2008n}
\bibliography{biblio/bibliography}

\appendix
\section*{Исходный код}
% \lstinputlisting{code/main.py} % пример вставки кода

\end{document}
