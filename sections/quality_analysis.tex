% Продолжая качественный анализ фреймворков, было предложено автором оригинального сравнения % // TODO добавить ссылку на оригинальное сравнение


\subsection{Популярность}


\begin{table}
    \centering
    \caption{Сравнение популярности}
    \label{tab:popularity-comparison}
    \begin{tabular}{lccc}
        \toprule
        \textbf{Фреймворк} & \textbf{Количество проектов} \\
        \midrule
        EML & 333 \\
        MLX & 35 \\
        TYXML & 80 \\ % // TODO добавить количество проектов
        DHTML & 70 \\
        \bottomrule
    \end{tabular}
\end{table}

Результаты сравнения популярности приведены в таблице \ref{tab:popularity-comparison}.
% https://github.com/search?q=dream+eml+language%3AOCaml+NOT+path%3Aexample&type=code


\subsection{Покрытие}

% сравнить генерацию покрытия при тестировании с использованием bisect\_ppx как стандартного инструмента.
Визуальное сравнение приведено в приложении 1 % // TODO добавить приложение
Его результаты таковы:

\begin{table}[h]
    \centering
    \caption{Сравнение покрытия}
    \label{tab:coverage-comparison}
    \begin{tabularx}{\linewidth}{l>{\raggedright\arraybackslash}X>{\raggedright\arraybackslash}XcXX}
        \toprule
        \textbf{Фреймворк} & \textbf{Комментарии} \\
        \midrule
        EML & \cellcolor{yellow!30} Изначальный шаблон теряется после работы препроцессора, результирующий код нечитабелен \\
        MLX & После препроцессора шаблон трансформируется в корректный OCaml код \\
        TYXML, DHTML & Не используется препроцессор, оригинальный код покрывается без проблем \\
        TYXML\% & Препроцессор в этом случае генерирует AST, которое в свою очередь обходится bisect-ом. Оригинальное представление кода сохраняется в покрытии \\
        \bottomrule
    \end{tabularx}
\end{table}

\subsection{Экосистема, дебаггинг и тестирование}

% // TODO доделать описание

\subsection{Документация}

% // TODO доделать описание
