Актуальность исследования обусловлена растущими требованиями к производительности веб-приложений.
Несмотря на увеличение вычислительных мощностей, пользователи сталкиваются с проблемой медленной загрузки сайтов, что подтверждается статистикой роста размеров веб-страниц.
% // TODO: добавить библиографию

% \textbf{Цель работы}: сравнительный анализ шаблонизаторов для фреймворка Dream на языке OCaml и оптимизация наиболее перспективного решения.

% \textbf{Задачи}:
% \begin{enumerate}
%     \item Анализ современных подходов к серверному рендерингу (SSR)
%     \item Сравнение характеристик шаблонизаторов: ocaml-mlx, tyxml, dream-html, dream-eml
%     \item Разработка методики тестирования производительности
%     \item Оптимизация механизма рендеринга dream-eml
%     \item Валидация результатов через нагрузочное тестирование
% \end{enumerate}

% \textbf{Научная новизна}: предложен механизм снижения аллокаций памяти в шаблонизаторах с сохранением чистоты функций.

Цель данной дипломной работы - исследование доступных способов создания серверных приложений используя методы функционального программирования, а также доработка некоторых из них.

\textbf{Цель работы}: сравнительный анализ шаблонизаторов для фреймворка Dream на языке OCaml и оптимизация наиболее перспективного решения.

Исходя из цели, в дипломной работе поставлены и решены следующие задачи:

\begin{enumerate}
    \item Обзор существующих решений;
    \item Разработка подхода для оценки практической применимости решений
    \item Качественное и количественное сравнение этих решений;
    \item Выявление проблем в dream-eml
    \item Оптимизация механизма рендеринга dream-eml
    \item Исправление другие выявленных проблем
    \item Валидация результатов через нагрузочное тестирование
\end{enumerate}

% Предметом исследования явилась совокупность 

Фреймворком dream пользуются большое количество проектов.
Пакетный менеджер языка OCaml называющийся opam не отслеживает количество скачиваний пакетов. Поэтому использовался ручной метод.
Самый простой способ вычислить количество проектов, исползующих dream - воспользоваться поисковиком github.
Он позволяет найти все упоминания фреймворка dream в .opam файлах, с результирующим количеством около 4000.
Из них пользуются:

\begin{enumerate}
    \item TyXML - 4000
    \item Dream eml - ~100
    \item XML - 2000
    \item html - хз пока
\end{enumerate}

% // TODO: докинуть ссылки на гитхаб и спросить можно ли так сделать у подлеса

Как видно, этими решениями пользуются, ондако dream eml несмотря на то что он поставляется вместе с dream пользуются меньше всего.
Хотелось бы это исправить

% // TODO
%///////////////////////////////// GENERATED BY DEEPSEEK WILL NEED REWRITING

\section*{Обоснование выбора Dream и Dream EML}
Несмотря на субъективный фактор инициализации исследования (рекомендация научного руководителя), выбор фреймворка Dream и шаблонизатора Dream EML для глубокого анализа обусловлен следующими объективными критериями:

\begin{enumerate}
    \item \textbf{Репрезентативность экосистемы OCaml}:
          Dream является \textit{де-факто} стандартом для веб-разработки на OCaml, что подтверждается:
          \begin{itemize}
              \item Наличием >2,000 проектов на GitHub, использующих Dream (по данным GitHub Advanced Search)
              \item Интеграцией с ключевыми инструментами OCaml-экосистемы (Dune, Opam, Lwt)
              \item Активной поддержкой сообщества (более 1,200 звёзд на GitHub)
          \end{itemize}

    \item \textbf{Архитектурная уникальность Dream EML}:
          Шаблонизатор представляет научный интерес благодаря:
          \begin{itemize}
              \item Гибридной модели (HTML-подобный синтаксис + полноценная интеграция с OCaml)
              \item Гарантиям чистоты функций, что соответствует принципам функционального программирования
              \item Минималистичной реализации (всего $\sim$400 LOC), удобной для анализа и модификации
              \item Отсутствии привязанности к рендерингу HTML-синтаксиса\footnote{Этот аспект не будет исследован в этой работе далекк качественного сравнения, однако в dream-eml есть возможность генерировать произвольные тексты также с inline-вставками OCaml кода}
          \end{itemize}

    \item \textbf{Неисследованность проблемы}:
          Экспериментальный анализ выявил \textit{критический пробел}:
          \begin{itemize}
              \item Отсутствие сравнительных исследований шаблонизаторов OCaml в академической литературе
              \item Документально подтверждённые дефициты Dream EML в производительности (до 4$\times$ медленнее аналогов)
              \item Проблемы интеграции с инструментами метрики качества (Bisect\_ppx)
          \end{itemize}

    \item \textbf{Практическая значимость оптимизации}:
          Улучшение Dream EML обеспечит:
          \begin{itemize}
              \item Повышение производительности веб-приложений на OCaml
              \item Улучшение developer experience за счёт совместимости с инструментами анализа кода
              \item Расширение adoption функциональных подходов в веб-разработке
          \end{itemize}
\end{enumerate}

\textit{Таким образом, фокус на Dream EML продиктован не только доступностью экспертизы, но и его уникальной позицией в экосистеме OCaml, наличием нерешённых исследовательских задач и высоким потенциалом для практического воздействия.}
% // TODO актуализировать числа
%///////////////////////////////////// GENERATED BY DEEPSEEK WILL NEED REWRITING

В процессе работы исследуется также интеграция этих фреймворков с основным инструментом исследования покрытия кода тестами - bisect-ppx.
Его выбор обоснован популянростью, также рекомендацией научного руководителя % // TODO переписать этот кусок и возможно перенести его


\textbf{Научная новизна} предложен механизм снижения аллокаций памяти в шаблонизаторах с сохранением чистоты функций.

% // TODO докинуть скриншоты и информацию о том как я генерировал bisect-ppx. Также, упомянуть проблему с потереей репрезентации

% // TODO докинуть html_of_jsx и в целом более подробно изучить какие еще есть варианты

