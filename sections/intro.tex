Развитие веб-разработки за последние три десятилетия было отмечено сменой архитектурных парадигм.
Сначала доминировали статические HTML-страницы, затем возникла динамическая генерация на стороне сервера (CGI, PHP, ASP), а с 2010-х годов широкое распространение приобрели одностраничные приложения (SPA, Single-page application) с преимущественно клиентским рендерингом \cite{Fowler2004, Osmani2017}.
Однако избыточное использование JavaScript и усложнение фронтенд-архитектуры привело к ухудшению пользовательских метрик — увеличению времени загрузки (Time to Interactive), "белым страницам" при запуске, снижению показателей поисковой оптимизации (SEO) и потере части мобильной аудитории \cite{GoogleLighthouse2021, HTTPArchive2022}.

Современные требования к производительности и SEO диктуют возврат к серверному рендерингу (SSR, Server-side rendering).
Такой подход позволяет сократить время ожидания первого контента, улучшить Core Web Vitals и обеспечить корректную индексацию страниц поисковыми системами \cite{ShopifyHydrogen2021}.
Примеры успешных гибридных фреймворков — Next.js, Nuxt, Remix — демонстрируют востребованность SSR в реальных проектах.

Тем не менее, большинство популярных решений для SSR ориентированы на императивные или объектно-ориентированные языки программирования.
В то же время всё большее внимание со стороны промышленности и исследовательского сообщества привлекает мультипарадигмальный язык программирования OCaml, сочетающий функциональный и императивный стили.
Особенностями OCaml являются высокая производительность благодаря нативной компиляции в C, строгая статическая типизация, развитая система модулей и выразительные абстракции, присущие функциональному программированию.
Согласно данным сайта Stack Overflow Developer Survey \cite{StackOverflow2022, StackOverflow2022}, популярность OCaml растёт: с 2022 по 2024 год процент разработчиков использующих OCaml увеличился в 1.5 раза.
Благодаря этим свойствам OCaml становится привлекательной платформой для построения производительных и надёжных серверных веб-приложений.

Согласно современным учебникам по веб-разработке, от эффективных шаблонизаторов ожидается поддержка переменных, достаточный уровень безопасности за счёт автоэкранирования данных, возможность реализовать условные логические конструкции в шаблонах, а также расширяемость за счёт интеграции компонентов и плагинов~\cite{Osipov2022}.
В данной работе именно эти критерии легли в основу сравнительного анализа рассматриваемых решений, а оценка эффективности проводится с точки зрения удобства для прикладного разработчика и практических требований промышленной разработки.

\subsection*{Цель и задачи исследования} Целью данной работы является сравнительный анализ и доработка шаблонизаторов для фреймворка Dream на языке OCaml с акцентом на выявление и устранение узких мест в производительности и удобстве использования.
В соответствии с поставленной целью в работе решаются следующие задачи:
\begin{enumerate}
    \item Провести обзор существующих решений для генерации HTML-разметки на сервере в экосистеме OCaml.
    \item Разработать формальные критерии оценки применимости и удобства различных шаблонизаторов для задач промышленной разработки.
    \item Провести их качественное и количественное сравнение по ключевым метрикам (производительность, удобство разработки, покрытие тестами).
    \item Исправить другие выявленные недостатки в рамках выбранного шаблонизатора.
    \item Верифицировать результаты посредством нагрузочного тестирования.
\end{enumerate}

\subsection*{Краткое описание предметной области}
Фреймворк Dream является одним из самых популярных инструментов для создания серверных приложений на OCaml благодаря простоте настройки, богатой документации и интеграции с современными средствами разработки.
Тем не менее, входящий в его состав шаблонизатор dream-eml активно используется значительно реже в сравнении с альтернативами, что обусловлено рядом технических причин, в том числе ограниченной функциональностью и отсутствием ряда оптимизаций.
В рамках дипломной работы этот дисбаланс планируется нивелировать за счёт внедрения современных приёмов статического анализа, повышения удобства интеграции с экосистемой OCaml и оптимизации производительности dream-eml.

\subsection*{Обоснование инструментов и научная новизна}
Процесс сравнительного анализа будет включать интеграцию фреймворков с инструментом bisect-ppx для оценки покрытия кода тестами, что повысит объективность сравнения.
Популярность данного решения и рекомендации научного руководителя обусловили его выбор.

\textbf{Научная новизна работы} заключается в предложении механизма снижения аллокаций памяти и оптимизации шаблонизации без потери чистоты функций, что обеспечивает баланс между производительностью и принципами функционального программирования.
