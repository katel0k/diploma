В результате работы были исследованы существующие подходы для генерации HTML страниц в рамках экосистемы OCaml.
Была разработана система для сравнения решений по качественным и количественным параметрам.

В результате проведенного сравнения, были выявлены следующие особенности сравниваемых фреймворков:

\begin{itemize}
    \item MLX - современное активно развивающееся решение, предоставляющее высокую производительность, поддержку современных методов разработки и удосбтво использования.
    \item TyXML - гигант индустрии, однако являющийся слишком большим и сложным для меньших проектов. Наблюдаются также проблемы с поризводительностью. Однако, для крупных проектов статическая типизация и валидации HTML становятся незаменимыми.
    \item TyXML\% - расширение TyXML, обладающее всеми теми же преимуществами, но с дополнительным синтаксисом. Отсутсвие синтаксической подсветки и недостаток документации не позволяют рассматривать это его современное решение
    \item Dream HTML - более простая альтернатива TyXML, однако не являющяся его заменой. Сценарии для его использования не были найдены, так как есть более подходящие альтернативы.
\end{itemize}

Основная работа была нацелена на Dream EML - препроцессор, являющийся встроенным решением для фреймворка Dream.
В результате сравнения было выделено множество его недостатков: низкая производительность с большим потрблением памяти, пробелы в документации, сложности в тестировании, а также отсутсвие синтаксической подсветки.
Его же уникальные свойства, в частности, простота в использовании а также уникальный строчный подход для генерации HTML, побудили попытку исправить эти недостатки.
В результате, они были систематически устранены:
\begin{itemize}
    \item Объем выделяемой памяти и вместе с тем скорость работы были улучшены посредством внедрения пула из буферов для генерируемых строк. Объем major аллокаций снизился с нескольких мегабайтов во многих случаях до нуля, что сделало его наиболее эффективным решением по используемой памяти.
    \item Документация была доработана, включив в себя более современный подход для настройки препроцессинга
    \item Был изменен формат результирующих после препроцессинга файлов - вместо сжатых строк теперь генерируются многострочные отчеты
    \item Была создана конфигурация синтаксической подсветки в VSCode, а также в отличие от остальных фреймворков были добавлены сниппеты
\end{itemize}

Таким образом, можно выделить 3 случая применения этих решений с соответсвующими рекомендациями от автора:
\begin{itemize}
    \item для SPA или простого сервера на Dream, Dream EML является простым и быстрым в настройке решением, которое тем не менее предоставляет(по результатам работы) все удобства промышленных фреймворков
    \item для современных проектов среднего размера MLX выглядит как наиболее подходящее решение в связи с его скоростью, активным развитием и удобством
    \item для крупной промышленной разработки - TyXML как установившееся решение в индустрии, предоставляющее кроме всего прочего валидацию HTML и экстенсивную поддержку сообщества
\end{itemize}
