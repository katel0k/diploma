\subsection{Проблема производительности веб-приложений}
Современные SPA\footnote{Single-page application}-приложения столкнулись с фундаментальным ограничением: необходимость выполнения значительного объема JavaScript-кода на стороне клиента перед отображением контента. % // TODO: докинуть библиографию
Это приводит к:
\begin{itemize}
    \item Увеличению времени полной загрузки (TTI)
    \item Проблемам с SEO\footnote{Search Engine Optimization}-индексацией
    \item Низкой производительности на мобильных устройствах
\end{itemize}

% // TODO: убедиться что я правильно понимаю кто такой сср, мало ли он позволяет прям хтмл реендирть на странице а не просто текст генерит

По информации издания CNews, первое решение на основе этой технологии было представлено в 2001 году. % https://www.cnews.ru/news/line/tehnologii_razrabotki_vebservisov
Это решение создает веб страницы на стороне сервера, что имеет следующие преимущества:
\begin{itemize}
    \item Мгновенную отдачу контента
    \item Улучшение Core Web Vitals
    \item Полную SEO-совместимость
    \item Кеширование результатов
    \item Сокрытие исходного кода
\end{itemize}

Сама проблема была вызвавна чрезмерным использованием веб-фремйворков, среди которых наиболее популярен react.js, который в свою очередь также продвигает SSR.

\subsection{Архитектурные принципы React}
React.js доминирует в веб-разработке благодаря:
\begin{itemize}
    \item Компонентной модели на основе чистых функций
    \item JSX как декларативному языку разметки
    \item Одностороннему потоку данных
\end{itemize}

В связи с этими факторами было разработано решение для разработки веб-серверов - dream. Для него было создано несколько способов генерировать html. 

% \subsection{Выбор OCaml для SSR}
% Обоснование использования OCaml:
% \begin{itemize}
%     \item \textbf{Типобезопасность}: статическая проверка шаблонов
%     \item \textbf{Функциональная парадигма}: естественная реализация React-подобных систем
%     \item \textbf{Производительность}: нативные бинарники через компиляцию в C
%     \item \textbf{Экосистема}: наличие фреймворка Dream для веб-разработки
% \end{itemize}




% \begin{lstlisting}[caption=Пример компонента на OCaml]
% let greet name = 
%   Dream.html (Dream.eml %s{<h1>Hello, <%s name %>!</h1>})
% \end{lstlisting}
