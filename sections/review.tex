Обзор существующих решений для генерации HTML-разметки в экосистеме OCaml выявил ряд подходов, различающихся архитектурными решениями, а также другими особенностями.
Основными объектами сравнения стали следующие фреймворки:

\textbf{Dream EML} - (далее EML) официальное решение предложенное авторами фреймворка Dream.
Реализуюет строковый подход к генерации HTML через трансляцию специализированного DSL в валидный OCaml-код.
Главными особенностями EML являются: минималистичная архитектура без внешних зависимостей, строгое соблюдение принципа чистоты функций и компиляция шаблонов в runtime-код.

\textbf{MLX} (октябрь 2023) - представляет современный гибридный подход, комбинирующий JSX-синтаксис с преобразованием через AST в OCaml-код (используя html\_of\_jsx).
Сейчас этот проект активно развивается, так на протяжении написания этой работы появилась поддержка в VSCode.
Однако ювенальный статус проекта, а также малая группа заинтересованных разработчиков не позволяют рассматривать его как стабильное решение.

\textbf{TyXML} (Ocsigen, 2016) и его расширение \textbf{TyXML\%} реализуют типобезопасный подход.
Использует AST для представления DOM, а также осуществляет статическую валидацию тегов на уровне типов.
Несмотря на промышленное происхождение, сложность интеграции и избыточность для простых сценариев ограничивают область применения.

\textbf{Dream HTML} - (далее DHTML) - компромиссное решение, сочетающее: лёгкую интеграцию с экосистемой Dream, AST представление с проверкой сбалансированности тегов и простоту использования.

Сравнительный анализ по ключевым параметрам (Табл. \ref{tab:previous-analysis}) выявил фундаментальное противоречие:
решения с AST-представлением (TyXML, DHTML) обеспечивают лучшую инструментальную поддержку, но теряют в простоте и соответствии функциональной парадигме, тогда как EML, будучи наиболее «функционально-чистым», лишён преимуществ статического анализа \cite{}. % // TODO библиография

\begin{table}[H]
    \centering
    \begin{tabular}{lccccc}
        \toprule
        \textbf{Параметр} & EML & MLX      & TyXML   & TyXML\% & DHTML   \\
        \midrule
        Проверка тегов    & Нет & Да       & Да      & Да      & Да      \\
        Валидация         & Нет & Частич.  & Частич. & Частич. & Нет     \\
        Интеграция OCaml  & Да  & Да       & Да      & Да      & Да      \\
        Экранирование     & Да  & Да       & Да      & Да      & Да      \\
        Поддержка IDE     & Нет & Огранич. & Да      & Да      & Частич. \\
        \bottomrule
    \end{tabular}
    \caption{
        Сравнительная характеристика шаблонизаторов по параметрам: \\
        Проверка тегов - проверка что HTML теги корректны и закрыты; \\
        Валидация - проверка соответствию HTML5 стандарту, например, что тег title использует только в пределах тега head; \\
        Интеграция OCaml - наличие возможности вставлять OCaml код посреди шаблонов и наоборот; \\
        Экранирование - корректное форматирование символов, имеющих специальное представление в HTML; \\
        Поддержка IDE - поддержка LSP и подсветки синтаксиса
    }
    \label{tab:previous-analysis}
\end{table}

Представленный анализ выявил существенный пробел в существующих сравнительных исследованиях — отсутствие систематизированных критериев выбора шаблонизатора для OCaml-экосистемы.
Актуальность данного исследования определяется тремя практическими задачами:
\begin{itemize}
    \item Формализация параметров сравнения для помощи разработчикам в выборе инструмента
    \item Создание методических рекомендаций по интеграции шаблонизаторов
    \item Доработка EML как перспективного, но недостаточно развитого решения
\end{itemize}
Последующий анализ направлен на создание объективной основы для сравнительной оценки, где особое внимание будет уделено параметрам, критически важным для промышленного внедрения.
