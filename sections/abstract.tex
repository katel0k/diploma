Современная веб-разработка характеризуется, с одной стороны, возрастающими проблемами производительности, а с другой — популяризацией подходов, основанных на принципах функционального программирования, чему примером служит React.js
В настоящей работе рассматривается потенциал использования языка OCaml для создания серверных решений с генерацией HTML (SSR).
В экосистеме OCaml были проанализированы и сравнены несколько библиотек-шаблонизаторов как по качественным, так и по количественным параметрам, включая производительность.
Особое внимание уделено шаблонизатору Dream EML, который отличается концептуальной чистотой и тесной интеграцией с фреймворком Dream. В рамках работы Dream EML был существенно доработан: реализована система синтаксической подсветки для VSCode, оптимизирован механизм рендеринга и внесены дополнительные улучшения, направленные на повышение удобства и эффективности использования.
