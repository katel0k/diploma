

Фреймворки сравниваются по следующим параметрам:
\begin{itemize}
    \item Популярность
    \item Вычисление покрытия
    \item Экосистема
    \item Дебаггинг
    \item Тестируемость
\end{itemize}

% Популярность существующих решеиний - важный параметр сравнения поскольку чем более он популярен, тем проще получить поддержку, тем больше экосистема, тем больше гарантия отсутсвия багов.


Все фреймворки в работе устанавливаются с помощью системы менеджмента пакетов opam.
Однако, она не предоставляет статистику использования различных пакетов.
В связи с этим для оценки популярности будет использоваться статистика Advanced Search на сайте GitHub.
Этот инструмент ограничен в своих возможностях, поэтому будут сделаны следующие предположения:
\begin{itemize}
    \item Проект доступен через GitHub.
    \item Проект использует opam в качестве системы менеджмента пакетов.
    \item Проект явно указывает использование фреймворка в качестве зависимости.
\end{itemize}


Поскольку tyxml\% является лишь частью tyxml а не отдельным фреймворком, он включен в сравнение только как его часть.
Похожая проблема возникает с EML, поскольку он является частью dream.
В этом случае статистика ищется по ключевому слову eml.


Покрытие генерируется и измеряется с помощью bisect\_ppx.
Фреймворки сравниваются с помощью тестовой страницы.
Она включает в себя условные конструкции, циклы и простые строки.
Полный код страницы приведен в приложении % // TODO \ref{app:test-page}.

Экосистема, дебаггинг и тестируемость оцениваются вместе, поскольку эти характеристики похожи.
Для каждого фреймворка ищутся соответсвующие инструменты и сравниваются в степени поддержки.
Чем более популярен инструмент / считается стандартным решением - тем считается лучше.

Документации по этим шбалонизаторам также не оставляют желать лучшего, поэтому будут разработаны гайды по интеграции всех этих фреймворков в проекты.

Фреймворки также сравниваются по количественным показателям. В частности, рассмотриваются следующие величины:

\begin{itemize}
    \item Время рендеринга
    \item Времени компиляции
    \item Потребление памяти
\end{itemize}

Для этого создана страница, генерирующая произвольное количество похожих элементов.
Каждый фреймворк получает свою реализацию этой страницы, с единственным переменным параметром - количеством элементов.
Все тесты помещены в один исполняемый файл, выводящий в консоль время выполнения каждого рендера.
После чего результаты агрегированы с помощью python и размещены на графике.

Выдвигается гипотеза что результирующий HTML генерируется алгоритмом с асимптотикой $\mathcal{O}(n)$, где n - количество сгенерированных элементов.
По точкам затем строится линейная аппроксимация с помощью метода наименьших квадратов.
Запуск программы осуществляется несколько раз чтобы избежать неточностей связанных с операционной системой, после чего результаты усредняются.

Время рендера измеряется с помощью функции time описанной в % // TODO добавить описание в приложение

Потребление памяти измеряется с помощью функции memory\_usage описанной в % // TODO добавить описание в приложение

Исходя из результатов сравнения, будет доработан EML в направлениях где он проявил себя хуже конкурентов.

% // TODO: упомянуть возникшую проблему с тем как я генерировал страницу?
% // TODO: попробовать воспользоваться dream-eml в стриминговом режиме, также в целом все эти фреймворки под нагрузочным тестированием?
