Разработанная система оценки преследует несколько взаимосвязанных целей.
Во-первых, практическая - описание способов интеграции упомянутых инструментов в существующие проекты, а также создание общего метода для выбора шаблонизатора под конкретные задачи.
Во-вторых, методическая - разработка подхода к тестированию HTML-генераторов для функциональных языков программирования.
В-третьих, прикладная - выявление узких мест EML для дальнейшей доработки.

Для достижения этих целей вводится двухуровневая система метрик, объединяющая качественные показатели и количественные измерения фреймворков в сравнении друг с другом.

Качественные показатели включают в себя популярность, тестируемость, экосистема, возможности отладки, документация и поддержка UTF-8.
Количественные же измерения состоят из сравнительной оценки скорости рендеринга, потребления памяти, а также скорости компиляции.
Каждый из параметров описан далее.

\subsection{Качественные параметры}

\textbf{Популярность} существующих решеиний - важный параметр сравнения поскольку чем более оно популярно, тем проще получить поддержку, тем больше экосистема, тем больше гарантия отсутсвия багов.

Все фреймворки в работе устанавливаются с помощью системы менеджмента пакетов opam.
Однако, она не предоставляет статистику использования различных пакетов.
В связи с этим для оценки популярности будет использоваться статистика Advanced Search на сайте GitHub.
Этот инструмент ограничен в своих возможностях, поэтому будут сделаны следующие предположения:
\begin{itemize}
    \item Проект доступен через GitHub.
    \item Проект использует opam в качестве системы менеджмента пакетов.
    \item Проект явно указывает использование фреймворка в качестве зависимости.
\end{itemize}

Поскольку tyxml\% является лишь частью tyxml а не отдельным фреймворком, он включен в сравнение только как его часть.

Похожая проблема возникает с EML, поскольку он является частью dream и не является прямой зависимостью.
В этом случае статистика ищется по использованному бинарнику dream\_eml в файлах dune.

Кроме того, сравнивается количество звезд на соответсвующих репозиториях проектов.
Эта метрика отражает отношение сообщества к фреймворку, даже если он реже используется.
Здесь также для EML статистика не будет рассматриваться и TyXML\% рассматривается совместно с TyXML.

GitHub Advanced Search позволяет сохранять запросы с помощью ссылок, используемые запросы приведены в таблице \ref{tab:urls}
% // TODO ГОСПОДИ ЧЕРТОВЫ ССЫЛКИ ИСПРАВЬ ИХ
\begin{table}[h!]
    \begin{tabular}{lc}
        \toprule
        \textbf{Фреймворк} & \textbf{Запрос} \\
        \midrule
        EML & \url{https://github.com/search?q=dream_eml+path\%3Adune&type=code} \\
        MLX & \url{https://github.com/search?q=mlx+path%3A.opam&type=code} \\
        TyXML & \url{https://github.com/search?q=tyxml+path%3A.opam&type=code} \\
        DHTML & \url{https://github.com/search?q=dream-html+path%3A.opam&type=code} \\
        \bottomrule
    \end{tabular}
    \caption{Ссылки для запросов в GitHub Advanced Search}
    \label{tab:urls}
\end{table}

% EML 258 https://github.com/search?q=dream_eml+path%3Adune&type=code
% MLX 61 https://github.com/search?q=mlx+path%3A.opam&type=code
% TyXML 1200 https://github.com/search?q=tyxml+path%3A.opam&type=code
% DHTML 20 https://github.com/search?q=dream-html+path%3A.opam&type=code

\textbf{Покрытие} генерируется и измеряется с помощью bisect\_ppx.
Это стандартный инструмент в экосистеме OCaml.
Он создает отчет о покрытии кода тестами в формате HTML.
Примеры отчетов приведены в приложении % // TODO \ref{app:bisect-ppx}.

Фреймворки сравниваются с помощью тестовой страницы.
Чтобы продемострировать возможности каждого фреймворка, а также точность сгенерированного покрытия, она включает в себя условные конструкции, циклы и простые строки.
Полный HTML страницы приведен в приложении% // TODO \ref{app:test-page}.
Поскольку фреймворки отличаются по принципу обработки шаблонов, для каждого из них будет написана своя реализация этого функционала.

\textbf{Экосистема, дебаггинг и тестируемость} оцениваются вместе, поскольку эти характеристики похожи.
Для каждого фреймворка ищутся соответсвующие инструменты и сравниваются в степени поддержки.
Чем более популярен инструмент - тем считается лучше.


\textbf{Документация} каждого шаблонизатора оценивается по следующим параметрам: полнота, наличие примеров, актуальность и наличие описания принципа работы фреймворка.

\textbf{Поддержка UTF-8} проверяется напрямую - страница с русским текстом генерируется используя стандартные настройки.
Если это не сработало, производятся поиски в документации способа активировать произвольную кодировку или хотя бы UTF-8.

\subsection{Количественные параметры}

\textbf{Общий подход}

Для сравнения производительности шаблонизаторов создана страница, генерирующая некоторое количество похожих маленьких элементов.
Каждый фреймворк получает свою реализацию этой страницы.
Дабы избежать погрешностей связанных с операционной системой, все тесты помещены в один исполняемый файл и запуск программы осуществляется несколько раз.
Время измеряется с помощью функции time описанной в % // TODO добавить описание в приложение
После чего результаты агрегированы с помощью python и размещены на графике.
Выдвигается гипотеза что результирующий HTML генерируется алгоритмом с асимптотикой $\mathcal{O}(n)$, где n - количество сгенерированных элементов.
По точкам затем строится линейная аппроксимация с помощью метода наименьших квадратов.

Потребление памяти измеряется с помощью функции memory\_usage описанной в % // TODO добавить описание в приложение

% // TODO: упомянуть возникшую проблему с тем как я генерировал страницу?
% // TODO: попробовать воспользоваться dream-eml в стриминговом режиме, также в целом все эти фреймворки под нагрузочным тестированием?
