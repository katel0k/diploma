\subsection{Установка для тестирования}

В ходе экспериментальной части работы для проведения сравнительного анализа были выбраны несколько фреймворков, требующих предварительной установки и настройки окружения.
Управление зависимостями и пакетами осуществлялось с помощью стандартного пакетного менеджера OCaml — \texttt{opam}.
В качестве системы сборки и автоматизации установки всех необходимых компонентов был использован инструмент \texttt{dune}, который обеспечивает генерацию конфигураций для \texttt{opam} и служит стандартом для описания проектов на OCaml.
В качестве базового компилятора использовалась версия OCaml 5.3.0 (январь 2025).
Перечень всех необходимых библиотек и точных версий зависимостей приведён в приложении \ref{apx:versions}

\subsection{Сравнение}

Разработанная система оценки преследует несколько взаимосвязанных целей.
Во-первых, практическая - описание способов интеграции упомянутых инструментов в существующие проекты, а также создание общего метода для выбора шаблонизатора под конкретные задачи.
Во-вторых, методическая - разработка подхода к тестированию HTML-генераторов для функциональных языков программирования.
В-третьих, прикладная - выявление узких мест EML для дальнейшей доработки.

Для достижения этих целей вводится двухуровневая система метрик, объединяющая качественные показатели и количественные измерения фреймворков в сравнении друг с другом.

Качественные показатели включают в себя популярность, тестируемость, экосистема, возможности отладки, документация и поддержка UTF-8.
Количественные же измерения состоят из сравнительной оценки скорости рендеринга и потребления памяти.
Каждый из параметров описан далее.

\subsubsection{Качественные параметры}

\textbf{Популярность} существующих решеиний - важный параметр сравнения поскольку чем более оно популярно, тем проще получить поддержку, тем больше экосистема, тем больше гарантия отсутсвия багов.

Все фреймворки в работе устанавливаются с помощью системы менеджмента пакетов opam, однако она не предоставляет статистику использования различных пакетов.
В связи с этим для оценки популярности будет использоваться статистика GitHub Advanced Search.
Этот инструмент ограничен в своих возможностях, поэтому будут сделаны следующие предположения:
\begin{itemize}
    \item Проект доступен через GitHub.
    \item Проект использует opam в качестве системы менеджмента пакетов.
    \item Проект явно указывает использование фреймворка в качестве зависимости.
\end{itemize}

Поскольку TyXML\% является лишь частью TyXML а не отдельным фреймворком, он включен в сравнение только как его часть.

Похожая проблема возникает с EML, поскольку он является частью Dream и не является прямой зависимостью.
В этом случае статистика ищется по использованному бинарнику dream\_eml в файлах dune.

Кроме того, сравнивается количество звезд на соответсвующих репозиториях проектов.
Эта метрика отражает отношение сообщества к фреймворку, даже если он реже используется.
Здесь также для EML статистика не будет рассматриваться и TyXML\% рассматривается совместно с TyXML.

GitHub Advanced Search позволяет сохранять запросы с помощью ссылок, используемые запросы приведены в таблице \ref{tab:urls}

\begin{table}[h!]
    \centering
    \begin{tabularx}{\textwidth}{lX}
        \toprule
        \textbf{Фреймворк} & \textbf{Запрос} \\
        \midrule
        EML & \url{https://github.com/search?q=dream_eml+path\%3Adune&type=code} \\
        MLX & \url{https://github.com/search?q=mlx+path\%3A.opam&type=code} \\
        TyXML & \url{https://github.com/search?q=tyxml+path\%3A.opam&type=code} \\
        DHTML & \url{https://github.com/search?q=dream-html+path\%3A.opam&type=code} \\
        \bottomrule
    \end{tabularx}
    \caption{Ссылки для запросов в GitHub Advanced Search}
    \label{tab:urls}
\end{table}

\textbf{Покрытие} генерируется и измеряется с помощью bisect\_ppx.
Это стандартный инструмент в экосистеме OCaml, создающий отчет о покрытии кода тестами в формате HTML.
Фактически он является препроцессором, запускаемым как внешний инструмент через dune.
Фреймворки отличаются по принципу обработки шаблонов, однако для всех фреймворков добавление генерации покрытия достигается добавлением секции \texttt{(instrumentation (backend bisect\_ppx))} в соответствующее объявление в файле dune.

Сравнение происходит с помощью тестовой страницы.
Чтобы продемострировать возможности каждого фреймворка, а также точность сгенерированного покрытия, она включает в себя условные конструкции, циклы и другие конструкции, которые могут появиться при разработке HTML шаблона.
Полный код страницы приведен в репозитории \href{https://github.com/katel0k/dream_templaters_comparison}{доступным по этой ссылке}.

\textbf{Экосистема} - параметр описывающий интеграцию библиотеки в разработку.
Оценивается поддержка IDE.
Существующая оценка является очень краткой, без подробного описания существующих инструментов и критериев оценивания.
Кроме того, она давно не актуализировалась.
В связи с его растущей популярностью \cite{VSCode2025}, Visual Studio code взят в качестве IDE под вопросом.
Рассматриваются следующие интеграции: поддержка синтаксической подсветки и наличие сниппетов(шаблоны кода, представляют собой механизмы автоподстановки часто используемых конструкций).
Поддержка этого функционала для OCaml в VSCode в основном осуществляется с помощью VSCode-OCaml-platform.
Это расширение предоставляет способы интеграции произвольного синтаксиса на основании TextMate движка подсветки в VSCode.

% Вторым параметром выступает поддержка сообщества.
% Поскольку все вышеупомянутые проекты являются open-source, существование общего форума или, более современно, Discord-сервера является для них важным критерием успеха.

% \textbf{Поддержка современных стандартов}

% В данном случае имеется в виду поддержка JSX подобного синтаксиса
% в частности, фрагментов
% произвольных аргументов
% вызов функций как компонентов
% подстановка ml кода


\textbf{Документация} каждого шаблонизатора оценивается качественно.

Согласно исследованию\cite{boehm1975high} % // TODO библиография
10\% стоимости разработки уходит на чтение документации %, причем 
и 47\% времени разработки в случае поддержки существующего кода занято чтением документации \cite{fjeldstad1983application}.
Утверждается также что плохая или неактуальная документация - более дорогостоящая чем отсутсвие документации вовсе\cite{poston1984does}
% x\% вызваны незадокументированным поведением
% --- вызваны устаревшей актуальностью документации
% также разработчики % // TODO
% отзывались, что наличие примеров ускоряет разработку более чем в 2 раза.
В связи с этим, фреймворки сравниваются по следующим параметрам:

\begin{enumerate}
    \item Наличие примеров интеграций с экосистемой, в частности, с dune и с Dream. Особенно релевантно это в случае препроцессоров, запуск которых требует дополнительных настроек
    \item Наличие примеров работы. Важен не только пример кода, но и продемонстрированный результат его исполнения.
    \item Наличие объяснения принципа работы. Тогда как возможности библиотеки могут быть определены из .mli файлов, понимание принципа работы требует вчитывания в код, что значительно замедляет разработку. Некорректное или же неполное понимание принципа работы ведет к неправильному использованию и/или к багам.
    \item Актуальность по сравнению с последней версией библиотеки. Сравниваются только последние даты обновлений.
    \item Корректность. Поскольку не представляется возможным оценить корректность всех документаций во всем их объеме, оцениваются только случаи с некорректной документацией.
    \item Полноту, как и корректность оценить в полном объеме невозможно, потому рассматриваются только найденные случаи незадокументированного поведения.
    \item Удобство, которое оценивается наличием сайта или HTML/PDF версии документации. Документация в README.md также считается хорошей. Многие разработчики оставляют докумнтацию только в .mli файлах, что делает чтение и поиск релевантной информации затрудительным.
\end{enumerate}

\textbf{Поддержка UTF-8} очевидно необходима для русскоговорящего сообщества.
90\% пользователей используют эмодзи для общения в интернете \cite{george2023emoji}.
Другие специальные символы, иероглифы и прочее, доступны только с кодировкой UTF-8.
В связи с чем это является важным параметром современного фреймворка для веб-разработки.

Проверяется поддержка напрямую - страница с русским текстом генерируется используя стандартные настройки.

\subsubsection{Количественные параметры}

Язык программирования OCaml использует автоматическое управление памятью, основанное на механизме сборки мусора\cite{ocamlMemoryModel}.
Основу этой модели составляет двуступенчатый генерационный сборщик мусора с дополнительной поддержкой компактного представления данных.
В OCaml вся динамически размещаемая память делится на две основные области: молодое (minor heap) и старое (major heap) поколения\cite{doligez1989realisation}.
Все новые объекты первоначально размещаются в молодом поколении, и, если они переживают несколько циклов сборки мусора, перемещаются в старое поколение.
Такой подход позволяет быстро удалять короткоживущие объекты (типичные, например, для временных структур данных).
OCaml использует стоп-коллектор для молодого поколения и инкрементальный для старого, благодаря чему достигается высокий уровень производительности на практике.
Кроме того, статическая типизация языка позволяет эффективно различать значения с управляемым и неуправляемым размещением — большинство примитивов (например, числа) могут храниться вне кучи (на стеке или внутри других значений), что снижает давление на сборщик мусора.
Важно отметить, что память в OCaml не требует явного освобождения: сборка мусора происходит прозрачно для программиста.
Однако при работе с большими объёмами данных или активном создании временных структур может возникнуть дополнительная нагрузка на систему управления памятью, что требует оптимизаций при разработке высокопроизводительных приложений\cite{allocImpact2019}.

Стандартный инструмент в экосистеме OCaml для измерения выделения памяти - ocaml-benchmark.
Это библиотека для точного профилирования производительности и анализа ресурсов в программах на OCaml.
Она предоставляет инструменты для измерения не только времени выполнения, но и потребления памяти, включая отслеживание аллокаций, сборок мусора (GC) и изменения размера кучи.
Библиотека интегрируется с рантаймом OCaml, позволяя собирать детализированную статистику на уровне отдельных функций или блоков кода.
Для анализа памяти она регистрирует такие метрики, как: пиковое использование памяти,количество аллокаций за период и аккумулятивное потребление памяти.
Инструмент поддерживает многократные прогоны тестов для минимизации погрешностей и предоставляет API для автоматизации замеров, что делает его подходящим для сравнительного анализа и оптимизации ресурсоемких участков кода.

Для сравнения производительности шаблонизаторов создана страница, генерирующая некоторое количество похожих маленьких элементов.
Каждый фреймворк получает свою реализацию этой страницы.
TyXML и TyXML\% будут разделены для измерений, однако ожидается что результаты будут одинаковыми.

Кроме сравнения с помощью ocaml-benchmark, производится также более грубое сравнение с помощью прямого измерения времени работы программы.
Это измерение также встраивается в рантайм программы, и представляет собой обычный секундомер.
Запуск осуществляется 5 раз, после чего результаты агрегированы с помощью python, усреднены и размещены на графике.
Выдвигается гипотеза что результирующий HTML генерируется алгоритмом с асимптотикой $\mathcal{O}(n)$, где n - количество сгенерированных элементов.
Такой подход позволяет оценить влияние сборки мусора и выделения памяти на реальное время исполнения кода.
